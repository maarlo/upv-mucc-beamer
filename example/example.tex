% Aspectratio 16:9 is recommended
\documentclass[aspectratio=169]{beamer}

% Metadata of the presentation
\title{MUCC beamer theme}
\subtitle{Powered by \LaTeX}
\date{miercóles, 14 de septiembre de 2022}
\author[maarlo9]{Mario Aragonés Lozano <\texttt{maarlo9@teleco.upv.es}>}

% Load the MUCC theme
\usetheme{mucc}

%%%%%%%%%%%%%%%%%%%%%%%%%%%%%%%%%%%%%%%%%%%%%%%%%%%%%%%%%%%%%%%%%
% Have this if you'd like section slides
% \AtBeginSection[]{
%     \sectionframe
% }

% Have this if you'd like a toc highlighting the current section
\AtBeginSection[]{
    \begin{frame}{Overview}
        \tableofcontents[currentsection]
    \end{frame}
}
%%%%%%%%%%%%%%%%%%%%%%%%%%%%%%%%%%%%%%%%%%%%%%%%%%%%%%%%%%%%%%%%%

\begin{document}

% I guess you always want a titleframe
\titleframe

% Have this if you'd like a frame containing the
% table of content
\begin{frame}{Overview}
    \tableofcontents[hideallsubsections]
\end{frame}

% Start of the first section
\section{Fonts and layout}

\begin{frame}
    \frametitle{This is the frame title}
    \framesubtitle{With optional frame subtitle}
    Regular text in the body of the slide is black and rendered in Corbel.\\[.5cm]
    \begin{equation*}
        F(x|\mu,s) = \int_{-\infty}^x s^{-1}\left(1+e^{-\frac{v-\mu}{s}}\right)^{-2} e^{-\frac{v-\mu}{s}}\;\mathsf{d}v = \frac{1}{1+e^{-\frac{x-\mu}{s}}}
    \end{equation*}
    Emphasis can be added by using \textbf{bold} typeface, \textit{italic}, {\color{mucc-alert}colors} or {\color{mucc-alert}\textbf{\textit{any combination}}}.\\
    More about colors follows later.
\end{frame}

\begin{frame}
    \frametitle{The frame title is rendered in Small Caps}
    The official Powerpoint/Keynote templates have all titles in both ALL CAPS, \textbf{bold} and \underline{underline}.\\[.5cm]
    In my opinion, this combination is somewhat \underline{\textbf{AGGRESSIVE AND UNPLEASANT TO THE EYE}}.\\[.5cm]
    Instead, this theme makes use of \textsc{Small Caps} for all titles and subtitles
\end{frame}


\section{Lists and enumeration}

\begin{frame}
    \frametitle{Lists of items}
    This is how a list of unnumbered items looks:
    \begin{itemize}
        \item Item 1
        \item Item 2
        \item Item 3
    \end{itemize}
    \vspace{.25cm}
    Nested lists of items are possible too:
    \begin{itemize}
        \item Item 1
            \begin{itemize}
                \item Subitem a
                \item Subitem b
            \end{itemize}
        \item Item 2
            \begin{itemize}
                \item Subitem a
                \item Subitem b
                    \begin{itemize}
                        \item Subsubitem a
                        \item Subsubitem b
                    \end{itemize}
            \end{itemize}
    \end{itemize}
\end{frame}

\begin{frame}
    \frametitle{Lists of items}
    \framesubtitle{With optional frame subtitle}
    This is how a list of unnumbered items looks:
    \begin{itemize}
        \item Item 1
        \item Item 2
        \item Item 3
    \end{itemize}
    \vspace{.25cm}
    Nested lists of items are possible too:
    \begin{itemize}
        \item Item 1
            \begin{itemize}
                \item Subitem a
                \item Subitem b
            \end{itemize}
        \item Item 2
            \begin{itemize}
                \item Subitem a
                \item Subitem b
                    \begin{itemize}
                        \item Subsubitem a
                        \item Subsubitem b
                    \end{itemize}
            \end{itemize}
    \end{itemize}
\end{frame}

\begin{frame}
    \frametitle{Lists of items}
    This is how a list of numbered items looks:\\[.25cm]
    \begin{enumerate}
        \itemsep.5cm
        \item Item 1
        \item Item 2
        \item Item 3
        \item Item 4
        \item Item 5
    \end{enumerate}
\end{frame}


\section{Colors}

\begin{frame}
    \frametitle{Colors}
    \begin{itemize}
        \item The offical MUCC colors (in RGB) are part of the theme.
        \item The primary MUCC color is {\color{mucc-pink} mucc-pink}, and the secondary color is {\color{mucc-black} mucc-black}.
            \begin{center}
                \begin{minipage}[t]{.35\textwidth}
                    \begin{itemize}
                        \item {\color{mucc-pink}  \textbf{mucc-pink}}
                        \item {\color{mucc-red}    \textbf{mucc-red}}
                        \item {\color{mucc-green}   \textbf{mucc-green}}
                    \end{itemize}
                \end{minipage}%
                \begin{minipage}[t]{.35\textwidth}
                    \begin{itemize}
                        \item {\color{mucc-black} \textbf{mucc-black}}
                        \item {\color{mucc-alert} \textbf{mucc-alert}}
                    \end{itemize}
                \end{minipage}
                \vspace{.5cm}
            \end{center}
    \end{itemize}
\end{frame}

\begin{frame}
    \frametitle{Example of mucc-alert}
    \begin{theorem}
        There is no largest prime number.
    \end{theorem}
    \pause
    \begin{block}{Proof}
        \begin{enumerate}
            \item<2-| alert@2> Suppose $p$ were the largest prime number.
            \item<3-| alert@3> Let $q$ be the product of the first $p$ numbers.
            \item<4-| alert@4> Then $q+1$ is not divisible by any of them.
            \item<5-6| alert@5> But $q + 1$ is greater than $1$, thus divisible by some prime number not in the first $p$ numbers.\only<6>\qed
        \end{enumerate}
    \end{block}
\end{frame}

\begin{frame}
    \frametitle{Another example of mucc-alert}
    Blocks can be created for definition, proofs, examples, etc.
    \begin{block}{Regular block}
        This is an important message.
    \end{block}
    \vspace{.5cm}
    \pause
    A special kind of block is the \texttt{alertblock}:
    \begin{alertblock}{Alert!}
        This is a very important message.
    \end{alertblock}
\end{frame}


\section{Frame numbers}

\begin{frame}
    \frametitle{Frame numbers}
    \begin{itemize}
        \itemsep.25cm
        \item By default frame numbers are places on every frame.
        \item The frame number is always followed by the total number of frames.
        \item The theme option \texttt{noframenumber} removes frame numbers on all slides.
    \end{itemize}
\end{frame}

% End presentation with titleframe
\titleframe

\end{document}
